\subsection{TCP Connection setup}
\begin{Question}
	Which segments are the initial three-way handshake in the trace file? How do you find them?
\end{Question}
\begin{Answer}
	The initial handshake occurs in segments 1, 2 and 3.
	Segments 1 and 2 have SYN flag set, which identify them as connection initiators.
	Segment 3 is the ACK to the server's SYN in segment 2.
\end{Answer}

\begin{Question}
	What is the actual initial sequence number in each direction (in hexadecimal format)?
\end{Question}
\begin{Answer}
	Client $\rightarrow$ Server: \hex{b93c1f07}\\
	Server $\rightarrow$ Client: \hex{8661d89a}
\end{Answer}

\begin{Question}
	What is the value of the acknowledgement number in the SYN/ACK segment? How did \texttt{gaia.cs.umass.edu} determine that value?
\end{Question}
\begin{Answer}
	The ACK number is \hex{b93c1f08}.
	It represents the next sequence that it can receive and is one greater than this Client $\rightarrow$ Server initial sequence number.
\end{Answer}

\begin{Question}
	What are the values of the sequence number and the acknowledgment number in the third ACK segments in the three-way handshake? How did the client determine these values?
\end{Question}
\begin{Answer}
	The sequence number is \hex{b93c1f08} and is the segment requested by the server in segment 2.
	The ACK is \hex{8661d89b} and corresponds to an acknowledgment of the server's SYN request.
\end{Answer}

\begin{Question}
	How did the client and the server announce the maximum TCP payload size that they were willing to accept? What are the values and why did they choose these values?
\end{Question}
\begin{Answer}
	The SYN requests from both the client and server contain a TCP option specifying an MSS of 1460 bytes.
	This corresponds to the optimal ethernet packet size of 1500 bytes, less two 20 byte headers.
\end{Answer}

\begin{Question}
	Is there data sent in the SYN, SYN/ACK, and ACK segment?
\end{Question}
\begin{Answer}
	No.
\end{Answer}
