\section{Procedure}\label{sec:procedure}
\subsection{ARP Functions}
\texttt{ethernet-trace-1.pcap} was downloaded from the course lab website and opened in Wireshark. The trace contains traffic generated by a source requesting a large document from a host. The trace begins with ARP discovery, where the source requests the MAC address from an IP. Both the request and reply headers for the ARP messages are examined in depth.

\subsection{Analyzing IP frames}
\texttt{ethernet-trace-1.pcap} also contains an HTTP GET request to initiate the transfer of the large document. The contents and size of the IP header corresponding to the request are examined in detail.

\subsection{ICMP Functions: Ping \& Traceroute}
\texttt{ping-trace-1.pcap} contains traffic from ten ping requests between hosts on a local network. The round trip time (RTT) between two hosts is determined as the difference in wall clock time between the sending of a request and receiving its reply.

\texttt{tracert-trace-2.pcap} contains traffic from the \texttt{traceroute} command for a source and destination on the UVic network. \texttt{traceroute} sends requests to the destination with staggered time-to-live (TTL) values. If the requests expire before they reach the destination the intermediate router will send back an error message wrapped around the original request header. The order of intermediate routers can be determined by cross-referencing the sender IP of the error message with the original request's TTL.
