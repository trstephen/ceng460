\section{Discussion}\label{sec:discussion}
\subsection{Running WireShark}\label{sec:wireshark}
\textit{List at least three different protocols that appear in an unfiltered capture.}

After invoking \verb|wget http://www.google.ca| at the command line, WireShark fills with traffic.
The following protocols were observed:
\begin{itemize}
	\item DNS
	\item HTTP
	\item TCP
	\item TLSv1.2
\end{itemize}

\textit{How long did it take from the HTTP GET message being sent to the HTTP OK reply being received?}

This is determined by comparing the values of the Time field in the Captured Packets window of WireShark.
The value of Time corresponds to the elapsed time the packet was recorded since WireShark began its capture.
For this capture, GET was sent at +7.271199 ms and OK was received at +56.242508 ms.
Thus, the elapsed time is 48.971318 ms.

\subsection{Network tools}\label{sec:tools}
\textit{How many Ethernet interfaces are in your computer, how to determine it?}

The output of the \texttt{ifconfig} command indicates that this computer has 1 network interface, \texttt{eth0}.
The loopback interface \texttt{lo} is a virtual network interface and is only used for self traffic.
\lstinputlisting[language={}]{data/eth.txt}

\textit{How to turn down/up an Ethernet interface?}

Use the command \texttt{ifconfig <interface> up|down}.
For example, to turn up the ethernet interface invoke \texttt{ifconfig eth0 up}.

\textit{Ping 10 packets to two websites.
Compare the statistic results (packet loss, average round-trip time)}

\lstinputlisting[language={}]{data/ping1.txt}
\lstinputlisting[language={}]{data/ping2.txt}

Neither ping resulted in any packet loss.
The ping to \verb|www.uvic.ca| had a faster round trip time than the ping to \verb|www.google.com|.
This is expected because, presumably, the call to \verb|www.uvic.ca| does not have to leave the LAN whereas the call to \verb|www.google.com| must travel through the internet to reach the nearest Google server.
This could be verified by executing \texttt{tracert} to determine the packet travel route.

\subsection{Layered protocol}\label{sec:protocol}

\textit{Draw the structure of a HTTP GET packet.}

% Adapted from http://www.texample.net/tikz/examples/pera-model/

\begin{figure}[htpb]
	\centering
	\begin{tikzpicture}[
	scale=0.75,
	start chain=1 going below, 
	start chain=2 going right,
	node distance=1mm,
	protocol/.style={
		scale=0.75,
		on chain=2,
		rectangle,
		rounded corners,
		draw=black, 
		very thick,
		text centered,
		text width=6cm,
		minimum height=12mm,
	},
	layer/.style={
		scale=0.75,
		on chain=1,
		minimum height=12mm,
		text width=2cm,
		text centered
	},
	every node/.style={font=\sffamily}
	]
	
	% Layers
	\node [layer] (APP) {Application};
	\node [layer] (TPT) {Transport};
	\node [layer] (NET) {Network};
	\node [layer] (LINK) {Link};
	
	% Protocols
	\chainin (APP); % Start right of APP
	\node [protocol] (HTTP) {HTTP/GET};
	\node [protocol, continue chain=going below] (TCP) {TCP};
	\node [protocol] (IP) {IPv4};
	\node [protocol] (eth0) {Ethernet};
	
	\end{tikzpicture}
\end{figure}


\textit{In the provided trace (lab1-wget-trace.pacp), calculate the average overhead of all the packets from the server to the client (in percentage).}

First, it is necessary to determine the IPs of the server and client.
Clients will initiate a GET request and servers will give the response.
The HTTP GET request is sent from \texttt{142.104.81.181} to \texttt{74.125.129.94}.
Therefore, the filter \[\texttt{ip.src==74.125.129.94 and ip.dst==142.104.81.181}\] will isolate traffic from the server to the client.

Next, the frame length and data length for each frame are determined from WireShark.
The Capture window has a column, Length, which gives the frame length.
The TCP header \texttt{Len} gives the data length.
The difference between the two lengths is the length of the header.
These values are recorded, \emph{manually}, for each frame.

The average overhead is given by:
\begin{equation*}
	\text{\% Overhead} = { \sum \left( \text{frame length} -  \text{data length} \right) \over \sum \text{frame length} } \times 100
\end{equation*}
and yeilds a 6.78\% overhead for the recorded traffic.

\newpage
\textit{Which Ethernet header field tells the next higher layer protocol is IP? What value is it used?}

The header \texttt{Type: IP (0x0800)} indicates that the network layer is IP.

\textit{Which IP header field tells the next higher layer protocol is TCP? What value is it used?}

The header \texttt{Protocol: TCP (6)} indicates that the transport layer is TCP.
