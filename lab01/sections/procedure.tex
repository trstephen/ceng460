\section{Procedure}\label{sec:procedure}
\subsection{Getting familiar with WireShark}
Workstations in ELW A321 run a version of Scientific Linux and are connected to the University of Victoria LAN.
WireShark, a network analysis tool, is installed on the workstations and is able to intercept all traffic over a workstation's ethernet connection.
This portion of the lab generates simple traffic in order to become familiar with the WireShark's interface.

WireShark is started and set to listen to traffic on the workstation's ethernet connection.
A webpage is requested via the console using \texttt{wget}.
This minimizes additional callback traffic generated by the browser version of most webpages and makes it easier to associate captured traffic to console commands.

\subsection{Networking tools}
Unix has several useful tools for investigating and configuring network interfaces:
\begin{itemize}
	\item \texttt{ping} will generate traffic to send to a destination and display statistics for the transmission times
	\item \texttt{netstat} displays information about the local network interfaces and can display local network statistics
	\item \texttt{ifconfig} configures and controls a machine's network interface.
\end{itemize}

Each of these tools will be used to make conclusions about network traffic.
See Section \ref{sec:tools} for examples of the output.

\subsection{Layered protocol}
WireShark is capable of displaying the data in each layer of network protocol in a frame.
An HTTP request / response conversation is recorded in \textit{lab1-wget-trace.pacp} and is examined in this section.
Information from the headers in each layer is used to determine the data length.
Headers also indicate the type of parent protocol.
