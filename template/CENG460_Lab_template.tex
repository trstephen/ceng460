%----------------------------------------------------------------------------
%----------------------------------------------------------------------------
%				    	SETUP
%----------------------------------------------------------------------------
%----------------------------------------------------------------------------

\documentclass[11pt]{article}

%----------------------------------------------------------------------------
%			  	   PACKAGES
%----------------------------------------------------------------------------

%%%%%%%%%%%%%%%%%%%%%%%
% 	  Packages
%%%%%%%%%%%%%%%%%%%%%%%

%% Fonts and Symbols
%% --------------------------
\usepackage{
	amsmath,			% math operators
	amssymb,			% math symbols
	courier,			% better tt font for listings
	soul,				% strike through with \st{}
	url,				% embed urls in text
	xcolor,				% color!
%	xfrac,				% fancy fractions
}

% preserve default font for URLs
\renewcommand*{\UrlFont}{\rmfamily}		

%% Graphics
%% --------------------
\usepackage{
	graphicx,			% allows insertion of images
	subfigure,			% allows subfigures (a), (b), etc.
}				
\graphicspath{ {graphics/} }	% (graphicx) relative path to graphics folder				

%% Tables
%% --------------------------
\usepackage{
	booktabs,			% better tables, discourages vertical rulings
	multicol,			% allow multi columns
}

%% Layout Alteration
%% --------------------------
\usepackage{			
%	caption,			% line breaks in captions with \\
%	changepage,			% change margins for PARTS of pages with (adjustwidth)
	geometry,			% change the margins for specific PAGES
	parskip,			% disable indents
	rotating,			% sideways figures
	setspace,			% single, double spacing
}
\geometry{				% specify page size options for (geometry)
	a4paper, 			% paper size
	hmargin=1in,		% horizontal margins
	vmargin=1in,		% vertical margins
}	


%% Units
%% --------------------------
\usepackage{
	siunitx,			% has S (decimal align) column type
}
\sisetup{input-symbols = {()},  % do not treat "(" and ")" in any special way
	group-digits  = false, 	% no grouping of digits
%	load-configurations = abbreviations,
%	per-mode = symbol,
}

%% Misc
%% --------------------------
\usepackage{
	enumitem,			% better control of enumerations, descriptions, etc
	listings,			% source code import and display
}

% colors used in listings
\definecolor{mygreen}{rgb}{0,0.6,0}
\definecolor{mygray}{rgb}{0.5,0.5,0.5}
\definecolor{mymauve}{rgb}{0.58,0,0.82}
\definecolor{darkblue}{rgb}{0,0,0.4}

\lstset{ %
	language=C,						% the language of the code
	basicstyle=\footnotesize\ttfamily\singlespacing,% the size of the fonts that are used for the code
	numbers=none,                   % where to put the line-numbers
	numberstyle=\tiny\color{mygray},% the style that is used for the line-numbers
	stepnumber=1,                   % the step between two line-numbers. If it's 1, each line
	% 	will be numbered
	numbersep=5pt,                  % how far the line-numbers are from the code
	backgroundcolor=\color{white},  % choose the background color. You must add \usepackage{color}
	showspaces=false,               % show spaces adding particular underscores
	showstringspaces=false,         % underline spaces within strings
	showtabs=false,                 % show tabs within strings adding particular underscores
	frame=single,	                % box the code [single, none]
	rulecolor=\color{black},        % if not set, the frame-color may be changed on line-breaks
	% 	within not-black text (e.g. commens (green here))
	tabsize=2,                      % sets default tabsize to 2 spaces
	captionpos=b,                   % sets the caption-position to bottom
	breaklines=true,                % sets automatic line breaking
	breakatwhitespace=false,        % sets if automatic breaks should only happen at whitespace
	title=\lstname,                 % show the filename of files included with \lstinputlisting;
	% 	also try caption instead of title
	keywordstyle=[1]\bfseries\color{darkblue},    % keyword style for mnemonics
	keywordstyle=[2]\bfseries\color{violet},	% keyword style for . mnemonics
	commentstyle=\color{mygreen},   % comment style
	stringstyle=\color{mymauve},    % string literal style
	escapeinside={\%*}{*)},         % if you want to add a comment within your code
	morekeywords={*,...}           	% if you want to add more keywords to the set
}

%% References
%% --------------------------
\usepackage[backend=biber,style=ieee]{biblatex}
\addbibresource{CENG460_Lab_template.bib}

%----------------------------------------------------------------------------
%		     MACROS AND COMMANDS
%----------------------------------------------------------------------------

% Defines a new command for the horizontal lines, change thickness here
\newcommand{\HRule}{\rule{\linewidth}{0.5mm}} 

% override S column type with centered text column
\newcommand{\textcol}[1]{\multicolumn{1}{c}{#1}}


%----------------------------------------------------------------------------
%----------------------------------------------------------------------------
%				   DOCUMENT
%----------------------------------------------------------------------------
%----------------------------------------------------------------------------

\begin{document}

\begin{titlepage}

\center
 
% Header
\textsc{\LARGE University of Victoria}\\[1cm] 	% Name of your university/college
\textsc{\Large CENG 460}\\[0.5cm] 			% Major heading such as course name
\textsc{\large Computer Communication Networks}\\[0.5cm] 		% Minor heading such as course title


% Lab Title
\HRule \\[0.4cm]
{\huge \bfseries Lab 2 - Ethernet and IEEE 802.11}\\[0.2cm] % Title of your document
\HRule \\[1.5cm]
 
 
%Lab Instructor Details
\begin{minipage}{0.7\textwidth}
\begin{flushleft} 

\large\emph{Instructor:} \\
Dr. Lin \textsc{Cai} \\
\vspace{12 pt}
\emph{Teaching Assistant:} \\
Amir \textsc{Andaliby}

\end{flushleft}
\end{minipage}
~
%% No content here, but it keeps the alignment of the instructor/TA
%% box correct.
%% Consider revising.
\begin{minipage}{0.1\textwidth}
\begin{flushright} \large

\vspace{12 pt}

\end{flushright}
\end{minipage}\\[2cm]


% Lab members
\Large Tyler \textsc{Stephen}
\large V00812021	\\
A01 - B04\\[1.5cm] 


% Date
{\large 4 March, 2016}\\ % Date, change the \today to a set date if you want to be precise

% Logo
\begin{figure}[b]	 % put logo at bottom of the page
	\centering
	\includegraphics[scale=0.3]{UVic_logo}
\end{figure}

\end{titlepage}


\doublespacing
\section{Introduction}\label{sec:intro}
This lab will examine link layer traffic over Ethernet and IEEE 802.11 (wi-fi). Source and destination MAC addresses will be examined for Ethernet. The composition of frame subtypes in IEEE 802.11 will be identified and compared. The reliability of a IEEE 802.11 signal will be determined by examining a series of Data frames.

\section{Procedure}\label{sec:procedure}
\subsection{Getting familiar with WireShark}
Workstations in ELW A321 run a version of Scientific Linux and are connected to the University of Victoria LAN.
WireShark, a network analysis tool, is installed on the workstations and is able to intercept all traffic over a workstation's ethernet connection.
This portion of the lab generates simple traffic in order to become familiar with the WireShark's interface.

WireShark is started and set to listen to traffic on the workstation's ethernet connection.
A webpage is requested via the console using \texttt{wget}.
This minimizes additional callback traffic generated by the browser version of most webpages and makes it easier to associate captured traffic to console commands.

\subsection{Networking tools}
Unix has several useful tools for investigating and configuring network interfaces:
\begin{itemize}
	\item \texttt{ping} will generate traffic to send to a destination and display statistics for the transmission times
	\item \texttt{netstat} displays information about the local network interfaces and can display local network statistics
	\item \texttt{ifconfig} configures and controls a machine's network interface.
\end{itemize}

Each of these tools will be used to make conclusions about network traffic.
See Section \ref{sec:tools} for examples of the output.

\subsection{Layered protocol}
WireShark is capable of displaying the data in each layer of network protocol in a frame.
An HTTP request / response conversation is recorded in \textit{lab1-wget-trace.pacp} and is examined in this section.
Information from the headers in each layer is used to determine the data length.
Headers also indicate the type of parent protocol.

\section{Discussion}\label{sec:discussion}


\section{Conclusion}\label{sec:conclusion}
Restate your results.
What did you learn?

\section{Feedback}\label{sec:feedback}
\begin{enumerate}
	\item Use \texttt{tracert} to examine the routes traveled during the \texttt{ping} comparison in Section \ref{sec:tools}. Does this explain the difference in round trip times? Compare \texttt{.ca}, \texttt{.com}, \texttt{.co.uk} and \texttt{.cn} routes.
	\item Write a script to compute traffic overhead.
\end{enumerate}


\newpage
\printbibliography[heading=bibintoc,title={References}]

\end{document}
